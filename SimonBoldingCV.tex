% (c) 2002 Matthew Boedicker <mboedick@mboedick.org> (original author) http://mboedick.org
% (c) 2003-2007 David J. Grant <davidgrant-at-gmail.com> http://www.davidgrant.ca
% (c) 2008 Nathaniel Johnston <nathaniel@nathanieljohnston.com> http://www.nathanieljohnston.com
%
%This work is licensed under the Creative Commons Attribution-Noncommercial-Share Alike 2.5 License. To view a copy of this license, visit http://creativecommons.org/licenses/by-nc-sa/2.5/ or send a letter to Creative Commons, 543 Howard Street, 5th Floor, San Francisco, California, 94105, USA.

\documentclass[letterpaper,11pt]{article}
\newlength{\outerbordwidth}
\pagestyle{empty}
\raggedbottom
\raggedright
\usepackage[svgnames]{xcolor}
\usepackage{framed}
\usepackage{tocloft}
\usepackage{etoolbox}
\robustify\cftdotfill

\makeatletter
\renewcommand\@biblabel[1]{#1.}
\makeatother


%-----------------------------------------------------------
%For including multiple sections of references I need 
% to redo this using biblatex

%-----------------------------------------------------------
%Edit these values as you see fit

\setlength{\outerbordwidth}{3pt}  % Width of border outside of title bars
\definecolor{shadecolor}{gray}{0.75}  % Outer background color of title bars (0 = black, 1 = white)
\definecolor{shadecolorB}{gray}{0.93}  % Inner background color of title bars


%-----------------------------------------------------------
%Margin setup

\setlength{\evensidemargin}{-0.25in}
\setlength{\headheight}{0in}
\setlength{\headsep}{0in}
\setlength{\oddsidemargin}{-0.25in}
\setlength{\paperheight}{11in}
\setlength{\paperwidth}{8.5in}
\setlength{\tabcolsep}{0in}
\setlength{\textheight}{9.5in}
\setlength{\textwidth}{7in}
\setlength{\topmargin}{-0.3in}
\setlength{\topskip}{0in}
\setlength{\voffset}{0.1in}


%-----------------------------------------------------------
%Custom commands
\newcommand{\resitem}[1]{\item #1 \vspace{-2pt}}
\newcommand{\resheading}[1]{\vspace{8pt}
  \parbox{\textwidth}{\setlength{\FrameSep}{\outerbordwidth}
    \begin{shaded}
\setlength{\fboxsep}{0pt}\framebox[\textwidth][l]{\setlength{\fboxsep}{4pt}\fcolorbox{shadecolorB}{shadecolorB}{\textbf{\sffamily{\mbox{~}\makebox[6.762in][l]{\large
    \hspace{0.01in}#1} \vphantom{p\^{E}}}}}}
    \end{shaded}
  }\vspace{-5pt}
}
\newcommand{\ressubheading}[4]{
\begin{tabular*}{6.5in}{l@{\cftdotfill{\cftsecdotsep}\extracolsep{\fill}}r}
		\textbf{\hspace{-0.341in}#1} & #2 \\
		\textit{#3} & \textit{#4} \\
\end{tabular*}\vspace{-6pt}}
%-----------------------------------------------------------
\newcommand{\resminorheading}[2]{
\begin{tabular*}{6.5in}{l@{\cftdotfill{\cftsecdotsep}\extracolsep{\fill}}r}
		\textbf{\hspace{-0.341in}#1} & #2 \\
\end{tabular*}\vspace{-1.6pt}}

\begin{document}

\begin{center}
\begin{tabular*}{3in}{c@{\extracolsep{\fill}}r}
\textbf{\huge Simon Bolding} \\
\textbf{\today} \\
Ph.D. Candidate at Texas A\&M University \\ simonrbolding@gmail.com 
\end{tabular*}
\end{center}



%%%%%%%%%%%%%%%%%%%%%%%%%%%%%%
\resheading{Education}
%%%%%%%%%%%%%%%%%%%%%%%%%%%%%%
\begin{itemize}
\item[] \ressubheading{Texas A\&M University\vspace{0.05in}}{College Station, TX}{Ph.D. Nuclear Engineering}{2013--Current}

	\begin{itemize}

        \resitem{Projected Defense Date, September 2016}
		\resitem{\textbf{DOE Nuclear Energy University Program Fellowship}, 3 years}

		\resitem{Emphasis on hybrid deterministic-Monte Carlo transport methods with finite elements for thermal radiative transfer.  Developed in  C++ research code.}
		\resitem{GPA: 4.0/4.0, Advisor: Jim E. Morel, Expected Graduation: December 2016}

    %    \resitem{Committee: Prof. Peter Frazier (advisor), Prof. Steve Strogatz, Prof. Bart Selman, Dr. Zhong Wang}

	\end{itemize}

\item[]

	\ressubheading{Kansas State University\vspace{0.05in}}{Manhattan, KS}{M.S. Nuclear Engineering}{2011--2013}

	\begin{itemize}
	
		\resitem{Thesis on two applications of Monte Carlo simulations: design and
        proof of concept of a neutron spectrometer and validation of nuclear data with subcritical multiplicity experiments}	
\resitem{GPA: 4.0/4.0, Advisor: Ken Shultis}
	\end{itemize}
	\ressubheading{\vspace{-0.31in}}{}{B.S. Mechanical Engineering with a Nuclear
    Engineering Option}{2007--2011}
	\begin{itemize}
	           \resitem{Graduated Summe Cum Laude with a Physics minor}
               \resitem{GPA: 4.0/4.0}
	\end{itemize}

\end{itemize}

%%%%%%%%%%%%%%%%%%%%%%%%%%%%%%
\resheading{Technical Training}
%%%%%%%%%%%%%%%%%%%%%%%%%%%%%%
\begin{itemize}
\item[]
\resminorheading{Languages}{C++, Python, Fortran 90/77}
\item[]
\resminorheading{Programs}{ MCNP5/6, Matlab, Excel, Solidworks, \LaTeX}
\item[]
\resminorheading{Development Tools}{Unix,  TotalView, Visual Studio, Git, CMake, Valgrind, UML}
\item[]
\resminorheading{Relevant Coursework}{Deterministic \& Monte Carlo Transport, Finite Element Methods,}
\resminorheading{}{ Multiphysics Coupling, Statistics \& Uncertainty Quantification,}
\resminorheading{}{ Engineering Analysis, Finite Differences, Parallel Algorithms}
\resminorheading{Active Q Clearance}{}
\end{itemize}


%%%%%%%%%%%%%%%%%%%%%%%%%%%%%%
\resheading{Awards}
%%%%%%%%%%%%%%%%%%%%%%%%%%%%%%
	\vspace{-2pt}
	\begin{center}\begin{tabular*}{6.6in}{l@{\extracolsep{\fill}}r}
		\multicolumn{2}{c}{Department of Energy Nuclear Energy University Program Fellowship \cftdotfill{\cftdotsep}2012-2015}\\
        \multicolumn{2}{c}{ANS Graduate Scholarship  \cftdotfill{\cftdotsep} 2011, 2012}\\
        \multicolumn{2}{c}{Outstanding Senior of KSU MNE Department Class of 2011 \cftdotfill{\cftdotsep} 2011}\\
        \multicolumn{2}{c}{Sigma Pi Sigma - Physics Honor Society  \cftdotfill{\cftdotsep} 2011}\\
        \multicolumn{2}{c}{Alpha Nu Sigma - Nuclear Engineering Honor Society  \cftdotfill{\cftdotsep} 2011}\\
		\vphantom{E}
\end{tabular*}
\end{center}\vspace*{-16pt}
\clearpage
%%%%%%%%%%%%%%%%%%%%%%%%%%%%%%
\resheading{Experience}
%%%%%%%%%%%%%%%%%%%%%%%%%%%%%%
\begin{itemize}
\item[]
\ressubheading{Lawrence Livermore National Laboratory: WCI Physics Div.}{Livermore, CA}{Graduate Intern}{Summer 2015}
	\begin{itemize}
	\resitem{Development and testing of acceleration methods for iterative Implicit Monte Carlo (IMC) method.}
	\resitem{Implemented methods in standalone version of Kull, a production C++ IMC code.}
	\resitem{Applied some OpenMPI parallelization}
	\end{itemize}
\ressubheading{Los Alamos National Laboratory: CCS-2}{Los Alamos, NM}{Graduate Intern}{Summer 2014}
	\begin{itemize}
	\resitem{Extended steady-state neutronics hybrid-Monte Carlo research code to handle time-dependent, grey \textbf{thermal radiative transfer} problems}
	\resitem{Developed methodology for thermal radiation physics, implemented non-linear solution method, and learned \textbf{C++ templates} and software development tools such as CMake, OOP design patterns, and simple XML parsing}
	\end{itemize}
 \ressubheading{Los Alamos National Laboratory: XCP-7}{Los Alamos, NM}{Graduate Intern}{Summer 2013}
	\begin{itemize}
	\resitem{Determining a spatially dependent cost function in \textbf{MCNP6} for weight-independent \textbf{variance reduction techniques}}
	\resitem{Developed and \textbf{integrated code} in the MCNP6 source}
	\resitem{Understanding of first and second moments of tally scores for importance map and high-resolution CPU clock timing modules}
	\end{itemize}
\ressubheading{Los Alamos National Laboratory: XCP-7}{Los Alamos, NM}{Graduate Intern}{Summer 2012}
	\begin{itemize}
	\resitem{Perturbing nuclear data on an energy dependent basis to correct bias by \textbf{MCNP} for \textbf{subcritical multiplicity simulations}}
	\resitem{Wrote in-depth \textbf{Python} modules for modifying Data that \textbf{extended object-oriented framework} for nuclear data at LANL}
	\resitem{Applied \textbf{statistical sampling methods} for generating new data based on ENDF covariance matrices}
	\end{itemize}
\ressubheading{Kansas State University}{Manhattan, KS}{Graduate Research Assistant}{01/2012--05/2013}
	\begin{itemize}
	\resitem{The Semiconductor Materials and Radiological Technologies research group}
	\resitem{Perform MCNP simulations as needed by the group for design and optimization of detectors, using \textbf{Python} scripts to automate optimizations}
	\resitem{Modeling of a Neutron Spectrometer and detection of nuclear devices from a distance, requiring application and automation of  \textbf{variance reduction} techniques.}
	\end{itemize}
\ressubheading{Knolls Atomic Power Laboratory: Transport Methods}{Niskayuna,
NY}{Intern}{Summer 2011}
	\begin{itemize}
	\resitem{Self-guided benchmarking of deterministic transport code in early development stages in Nuclear Data and Methods Unit}
	\resitem{Gained a basic understanding of deterministic transport physics and numerical methods}
	\end{itemize}
    \clearpage
\ressubheading{Knolls Atomic Power Laboratory: Spent Fuel Analysis}{Niskayuna,
NY}{Intern}{Summer 2010}
\begin{itemize}	
	\resitem{Spent Fuel Analysis Unit.  Interacted with requestors to understand
scope and objective of project. Learned how to use in-house Monte Carlo code, similar
to MCNP, to develop computer models}
	\resitem{Used the models to study fundamental effects on reactivity and tied
basic principles of nuclear engineering to practical laboratory applications.
Self-initiated side-studies to understand unique conditions found during reactivity
studies}
	\resitem{Documented work in technical reports and provided a clear and organized
presentation to reactor physics community}
\end{itemize}
\ressubheading{Kansas State University: Standoff Bomb Detection Group}{Manhattan,
KS}{Undergraduate Researcher}{06/2009--05/2011}
\begin{itemize}
	\resitem{Project Goal was to build a prototype device for detecting improvised
explosives from a safe distance using backscattered radiation}
	\resitem{Performed \textbf{MCNP5} modeling of experiments for graduate students, creating
scripts to automate the making of input files. Performed neutron and photon backscatter
experiments, gamma ray spectroscopy, troubleshooting integration
software}
\end{itemize}

\end{itemize}


%%%%%%%%%%%%%%%%%%%%%%%%%%%%%%
\resheading{Publications}
\renewcommand\refname{\vspace{-0.457in}}
\bibliographystyle{plainyr-rev}
\bibliography{references}
\nocite{*}




\end{document}
